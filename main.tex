%%%Tools
%%%%%http://www.tablesgenerator.com/#
%%%%%https://www.codecogs.com/latex/eqneditor.php
%%%%%https://www.sharelatex.com/learn/List_of_Greek_letters_and_math_symbols
%%%%%https://oeis.org/wiki/List_of_LaTeX_mathematical_symbols
%%%%%https://tex.stackexchange.com/questions/58098/what-are-all-the-font-styles-i-can-use-in-math-mode

\documentclass[hyperref={pdfpagelabels=false}]{beamer}
\usepackage{subfiles}
\usepackage[utf8]{inputenc} %recongnize pt {ç, ã}

\usepackage{ragged2e}
\usepackage{lipsum}
\let\olditemize=\itemize 
\renewenvironment{itemize}{\olditemize\justify}{\endlist} 

\usepackage[backend=biber]{biblatex}
\bibliography{rl.bib}

% Removes icon in bibliography
\setbeamertemplate{bibliography item}{}

\usepackage{amsmath}
\usepackage{algorithm,algorithmic}

%\bibliographystyle{apalike}
%\bibliographystyle{plainnat}
    
% There are many different themes available for Beamer. A comprehensive
% list with examples is given here:
% http://deic.uab.es/~iblanes/beamer_gallery/index_by_theme.html
% You can uncomment the themes below if you would like to use a different
% one:
%%\usetheme{AnnArbor}
%\usetheme{Antibes}
%\usetheme{Bergen} %side bar
%\usetheme{Berkeley} %side bar
%\usetheme{Berlin} %bullet progress
%\usetheme{Boadilla} %clean
%\usetheme{boxes} %clean
%\usetheme{CambridgeUS} %very red
%\usetheme{Darmstadt}  %bullet progress side-by-side
%\usetheme{default}
%\usetheme{Frankfurt}
%\usetheme{Goettingen}
%\usetheme{Hannover}
%\usetheme{Ilmenau}
%\usetheme{JuanLesPins}
%\usetheme{Luebeck}
%\usetheme{Madrid}
%\usetheme{Malmoe}
%\usetheme{Marburg}
%\usetheme{Montpellier}
%\usetheme{PaloAlto}
%\usetheme{Pittsburgh}
%\usetheme{Rochester}
%\usetheme{Singapore}
%\usetheme{Szeged}
%\usetheme{Warsaw}
\usetheme{metropolis}

\title{Hierarchical Reinforcement Learning}
\subtitle{Exame de Qualificação do Doutorado}

\author{Renata Garcia Oliveira}


\institute[PUC-Rio]
{
  Departamento de Engenharia Elétrica\\
  Pontifícia Universidade Católica do Rio de Janeiro
}

\date{Exame de Qualificação, 2017.2}

\subject{Theoretical Reinforcement Learning}
% This is only inserted into the PDF information catalog. Can be left out. 

% If you have a file called "university-logo-filename.xxx", where xxx
% is a graphic format that can be processed by latex or pdflatex,
% resp., then you can add a logo as follows:

% \pgfdeclareimage[height=0.5cm]{university-logo}{university-logo-filename}
% \logo{\pgfuseimage{university-logo}}

% Delete this, if you do not want the table of contents to pop up at
% the beginning of each subsection:
\AtBeginSubsection[]
{
  \begin{frame}<beamer>{Outline}
    \tableofcontents[currentsection,currentsubsection]
  \end{frame}
}

% Let's get started
\begin{document}

\begin{frame}
  \titlepage
\end{frame}

\begin{frame}{Outline}
  \tableofcontents
  % You might wish to add the option [pausesections]
\end{frame}

% Section and subsections will appear in the presentation overview
% and table of contents.

% http://citeseerx.ist.psu.edu/viewdoc/download?doi=10.1.1.190.3128&rep=rep1&type=pdf
% https://web.eecs.umich.edu/~baveja/Papers/IMRLIEEETAMDFinal.pdf
% http://www.cs.bham.ac.uk/research/projects/cogaff/aiib/papers/invited.d/lewis-singh-barto-2010-aiib-mar-02.pdf


\subfile{sections/resumo.tex}

\subfile{sections/introducao.tex}

\subfile{sections/temporalAbstraction.tex}

\subfile{sections/hrl.tex}

\subfile{sections/sutton1999.tex}

\subfile{sections/kulkarni2016.tex}

\subfile{sections/secondsec.tex}
% Placing a * after \section means it will not show in the
% outline or table of contents.
\subfile{sections/sumary.tex}


%\begin{frame}{Reference}
    %\bibliographystyle{amsalpha}
    %\bibliographystyle{abbrvnat}
%    \bibliography{rl}
%\end{frame}


\begin{frame}[allowframebreaks]
\frametitle{References}
% This prints the bibliography on the slide
\printbibliography
\end{frame}

\end{document}


